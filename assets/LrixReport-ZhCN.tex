%
% Lrix
% Copyright (C) 2025 lrisguan <lrisguan@outlook.com>
% 
% This program is released under the terms of the GNU General Public License version 2(GPLv2).
% See https://opensource.org/licenses/GPL-2.0 for more information.
% 
% Project homepage: https://github.com/lrisguan/Lrix
% Description: A scratch implemention of OS based on RISC-V
%

\documentclass[12pt]{ctexart}
\usepackage[margin=1.9cm]{geometry}
\usepackage[UTF8]{ctex}
\usepackage{times}
\usepackage{hyperref}
\usepackage{longtable}
\usepackage{array}
\usepackage{listings}
\usepackage{xcolor}
\usepackage{amsmath}
\usepackage{amssymb}
\usepackage{fontspec}
\usepackage{graphicx}
\usepackage{subfigure}
\usepackage[backend=biber,style=numeric,doi=false,isbn=false,url=true]{biblatex}
\renewcommand{\UrlFont}{\rmfamily}

\addbibresource{./lrixref.bib}

\setmainfont{TeX Gyre Termes}
\setmonofont{DejaVu Sans Mono}
\makeatletter
\newcommand\dlmu[2][4cm]{\hskip1pt\underline{\hb@xt@ #1{\hss#2\hss}}\hskip3pt}
\makeatother

\title{\textbf{Lrix: 一个草稿(scratch) RISC-V 操作系统}}

\author{
  \kaishu \Large 项目: Lrix\\
  \Large GitHub: \url{https://github.com/lrisguan/Lrix} \\
  \Large Author: 官稚淇 \\
  }
\date{\today}

\begin{document}
\maketitle

\tableofcontents

\begin{abstract}
Lrix 是一个针对 RISC-V 架构和 QEMU \texttt{virt} 机器平台,从零开始编写的运行在机器模式
下的教学操作系统。这也是我为什么称它为 scratch的原因(也许在将来我会解决这个问题)。
本报告对内核与用户空间的设计进行了简要的技术概览,重点介绍了内存管理、进程管理、
文件系统、Trap 与中断处理、系统调用接口以及内置 Shell。
\end{abstract}

\section{引言}
Lrix 是一个使用 C 语言和 RISC-V 汇编实现的小型教学导向操作系统。它在 QEMU \texttt{riscv64 virt}
平台上启动,使用简单的 UART 驱动进行控制台 I/O,并提供了极简的类 UNIX 进程与文件抽象。
顶层的 \texttt{Makefile} 构建内核镜像和用户程序,并使用 VirtIO 块设备作为后端存储在 QEMU 中运行。
注:在开发 Lrix 时,我参考了 \cite{riscv}, \cite{riscv:priv}, \cite{riscv:user},
\cite{xv6-riscv}, \cite{xv6-riscv}, \cite{virtio}, \cite{virtio-1.1}。

内核入口点 \texttt{kmain} 执行以下高层级步骤:
\begin{enumerate}
  \item 初始化 UART 以进行串行输出。
  \item 初始化 Trap/中断子系统和 PLIC。
  \item 初始化物理页分配器和虚拟内存管理器。
  \item 初始化调度器并创建 Idle 进程。
  \item 初始化 VirtIO 块设备驱动和基于 inode 的简单文件系统。
  \item 创建运行 Shell 的初始用户进程,然后开启中断,并通过 \texttt{wfi} 进入空闲状态,
        由定时器中断驱动调度。
\end{enumerate}

\section{内核子系统}
本节总结了 \texttt{kernel/} 下的主要内核模块,重点介绍内存管理、进程管理、文件系统和 Trap 处理。

\subsection{引导与入口 (boot/)}

\texttt{boot/start.S} 中的引导代码负责底层的 CPU 初始化:
\begin{itemize}
  \item 将初始栈指针设置为链接器提供的 \texttt{\_stack\_top}。
  \item 通过从 \texttt{\_bss\_start} 迭代到 \texttt{\_bss\_end} 来清零 BSS 段。
  \item 调用 C 入口点 \texttt{kmain},如果该函数返回,则进入死循环。
\end{itemize}
这段代码在 Machine 模式下运行,为内核准备一个干净的 C 环境。

\subsection{UART 与日志 (uart/, include/log.h)}

\texttt{uart/uart.c} 中的 UART 驱动实现了针对 QEMU \texttt{virt} 机器上映射在
\texttt{0x10000000} 的兼容 NS16550 的 UART 的中断 I/O。(对于中断I/O,你可以开启
编译选项 \textbf{TRAP\_DEBUG} 来查看)。
它提供:
\begin{itemize}
  \item \texttt{uart\_init}: 将 UART 配置为 8N1 模式。
  \item \texttt{uart\_putc}, \texttt{uart\_getc}, 和 \texttt{uart\_getc\_blocking}:
        用于基本的字符 I/O。
  \item \texttt{uart\_getline} 和一个小型的类似 scanf 的 \texttt{scank}: 用于解析用户输入。
  \item \texttt{printk}: 一个极简的格式化打印例程,支持标准的整数、字符串和指针格式。
\end{itemize}
头文件 \texttt{include/log.h} 在 \texttt{printk} 和 ANSI 转义码的基础上构建,提供了
诸如 \texttt{INFO}, \texttt{WARNING}, 和 \texttt{ERROR} 等内核通用的彩色日志宏。

\subsection{字符串库 (string/)}

\texttt{string/string.c} 文件实现了内核和用户空间使用的一小部分 libc 字符串和内存函数,
包括 \texttt{memset}, \texttt{memcpy}, \texttt{memmove}, \texttt{memcmp},
\texttt{strlen}, \texttt{strcmp}, 和 \texttt{strncmp}。这避免了对外部 C 库的依赖。

\subsection{底层 RISC-V 辅助函数 (include/riscv.h, include/types.h)}

头文件 \texttt{include/riscv.h} 包含常用 RISC-V CSR 的内联汇编辅助函数,特别是
\texttt{intr\_on} 和 \texttt{intr\_off},它们用于设置或清除 \texttt{mstatus}
中的机器中断使能 (MIE) 位。这在调度器和 Trap 代码中被广泛用于划定临界区。

头文件 \texttt{include/types.h} 定义了基本类型和 \texttt{RegState} 结构体,
该结构体保存调度器和上下文切换机制使用的 RISC-V 寄存器上下文(通用寄存器、
\texttt{sepc}、\texttt{sp} 和 \texttt{mstatus})。

\newpage
\begin{longtable}{|c|l|l|p{0.55\linewidth}|}
\caption{寄存器状态结构体字段描述}
\label{tab:registerstate_en}                                   \\
\hline
Field Name & Register Alias & Field Type & Description         \\
\hline
\endfirsthead
\caption[]{寄存器状态结构体字段描述(续表)}                      \\
\hline
成员名 & 寄存器别名 & 字段类型 & 描述                             \\
\hline
\endhead
\hline
\endfoot
\hline
\endlastfoot
x1         & ra         & uint64\_t  & 返回地址(Return Address)\\
\hline
x5         & t0         & uint64\_t  & 临时寄存器0              \\
\hline
x6         & t1         & uint64\_t  & 临时寄存器1              \\
\hline
x7         & t2         & uint64\_t  & 临时寄存器2              \\
\hline
x8         & s0/fp      & uint64\_t  & 保存寄存器0/帧指针       \\
\hline
x9         & s1         & uint64\_t  & 保存寄存器1              \\
\hline
x10        & a0         & uint64\_t  & 参数/返回值寄存器0       \\
\hline
x11        & a1         & uint64\_t  & 参数/返回值寄存器1       \\
\hline
x12        & a2         & uint64\_t  & 参数寄存器2              \\
\hline
x13        & a3         & uint64\_t  & 参数寄存器3              \\
\hline
x14        & a4         & uint64\_t  & 参数寄存器4              \\
\hline
x15        & a5         & uint64\_t  & 参数寄存器5              \\
\hline
x16        & a6         & uint64\_t  & 参数寄存器6              \\
\hline
x17        & a7         & uint64\_t  & 参数寄存器7              \\
\hline
x18        & s2         & uint64\_t  & 保存寄存器2              \\
\hline
x19        & s3         & uint64\_t  & 保存寄存器3              \\
\hline
x20        & s4         & uint64\_t  & 保存寄存器4              \\
\hline
x21        & s5         & uint64\_t  & 保存寄存器5              \\
\hline
x22        & s6         & uint64\_t  & 保存寄存器6              \\
\hline
x23        & s7         & uint64\_t  & 保存寄存器7              \\
\hline
x24        & s8         & uint64\_t  & 保存寄存器8              \\
\hline
x25        & s9         & uint64\_t  & 保存寄存器9              \\
\hline
x26        & s10        & uint64\_t  & 保存寄存器10             \\
\hline
x27        & s11        & uint64\_t  & 保存寄存器11             \\
\hline
x28        & t3         & uint64\_t  & 临时寄存器3              \\
\hline
x29        & t4         & uint64\_t  & 临时寄存器4              \\
\hline
x30        & t5         & uint64\_t  & 临时寄存器5              \\
\hline
x31        & t6         & uint64\_t  & 临时寄存器6              \\
\hline
sepc       & -          & uint64\_t  & 异常返回地址             \\
\hline
sp         & -          & uint64\_t  & 栈指针(Stack Pointer)  \\
\hline
mstatus    & -          & uint64\_t  & 机器态状态寄存器         \\
\hline
\end{longtable}

\subsection{物理内存管理 (mem/kmem.c)}

物理内存分配由 \texttt{mem/kmem.c} 中的简单页式分配器执行。分配器由 \texttt{kinit}
初始化,后者接收链接描述文件中定义的堆范围。

该实现维护:
\begin{itemize}
  \item 一个位于堆起始位置的 \texttt{Page} 描述符数组。
  \item 一个 \texttt{Page} 对象的单向空闲链表。
  \item 总页数、空闲页数和已用页数的计数器。
\end{itemize}
关键操作包括:
\begin{itemize}
  \item \textbf{初始化}: \texttt{Page} 描述符数组占用堆起始处的整数个页面;
                        这些页面被标记为已用,其余页面插入空闲链表。
  \item \textbf{分配 (kalloc)}: 从空闲链表中弹出一个 \texttt{Page},将其标记为已用,
                            根据其索引计算物理地址,并在返回给调用者之前将该页面清零。
  \item \textbf{释放 (kfree)}: 验证地址和对齐方式,将其映射回 \texttt{Page} 描述符,
                              防止双重释放,并将其推回空闲链表。
\end{itemize}
诊断辅助函数暴露统计信息(总/空闲/已用页数)并可打印分配器状态摘要。

\subsection{虚拟内存管理实现(mem/vmm.c)}
虚拟内存管理器(Virtual Memory Manager, VMM)实现了标准的RISC-V Sv39页表层级结构,
负责内核基础映射(内存RAM和内存映射I/O(MMIO))以及用户堆页面的按需分配。

\paragraph{Sv39页表布局}
Lrix采用标准的三级Sv39页表方案:
\begin{itemize}
  \item 虚拟地址分为VPN2(38-30位)、VPN1(29-21位)、
        VPN0(20-12位)和4KB页偏移量(11-0位)。
  \item 每个页表页大小为4KB,包含512个64位页表项(PTE)。
  \item 叶子页表项(Leaf PTE)编码物理页号(PPN)以及Sv39标志位(V/R/W/X/U/A/D)。
  \item 非叶子页表项(中间层级)标记为有效(V=1),但清除所有R/W/X/A/D位,
        这符合Sv39规范要求。
\end{itemize}
内核维护一个单独的二级根页表(level-2),名为\texttt{kernel\_pd},
其物理地址存储在\texttt{kernel\_pd\_phys}中。
所有页表内存均通过物理内存分配器(\texttt{kalloc})分配,
并在使用前显式初始化为零。

\paragraph{初始化与恒等映射}
入口函数\texttt{vmm\_init}执行以下步骤:
\begin{itemize}
  \item 分配并清零二级根页表,将其虚拟地址和物理地址分别存储到
        \texttt{kernel\_pd}和\texttt{kernel\_pd\_phys}。
  \item 执行最小化自测试(\texttt{vmm\_self\_test}):将测试虚拟地址映射到
        新分配的物理页,通过\texttt{vmm\_translate}验证地址转换正确性,
        然后解除映射并确认转换失效。
  \item 为\texttt{virt}虚拟机的关键内存区域建立恒等映射(虚拟地址=物理地址):
        \begin{itemize}
          \item 主内存(Main RAM):\texttt{0x80000000}--\texttt{0x80000000 + 128MB},
                映射为可读可写且用户可访问(\texttt{VMM\_P\_RW | VMM\_P\_USER}),
                确保启用分页后内核和用户的代码/数据/栈仍能正常工作。
          \item UART MMIO(地址\texttt{0x10000000})和VirtIO MMIO窗口
                \texttt{[VIRTIO\_MMIO\_START, VIRTIO\_MMIO\_END)},
                映射为内核独占的可读可写区域。
          \item CLINT定时器(\texttt{0x02000000}--\texttt{0x02010000}),
                映射为内核可读可写。
          \item PLIC MMIO区域(起始地址\texttt{PLIC\_BASE}),
                为中断控制器寄存器分配小型恒等映射窗口。
        \end{itemize}
\end{itemize}
这些恒等映射确保启用Sv39地址转换后,内核仍可使用相同的虚拟地址访问代码、数据、
堆和设备寄存器。

\paragraph{启用Sv39分页机制}
初始化完成后,\texttt{kmain}调用\texttt{vmm\_activate}启用硬件分页:
\begin{itemize}
  \item 计算页表物理页号(PPN):\texttt{kernel\_pd\_phys >> 12}。
  \item 构造Sv39格式的\texttt{satp}寄存器值:MODE=8(表示Sv39模式)、ASID=0,
        低34位存储页表PPN。
  \item 将该值写入\texttt{satp}控制状态寄存器(CSR),并执行
        \texttt{sfence.vma}指令刷新过时的TLB条目。
\end{itemize}
为保证可移植性,保留了两个占位钩子函数\texttt{arch\_set\_cr3}和
\texttt{arch\_enable\_paging},但在RISC-V架构上不执行任何操作;
所有实际的分页启用逻辑均通过\texttt{satp}寄存器实现。

\paragraph{映射、解除映射与地址转换}
核心操作基于一个遍历三级页表的辅助函数实现:
\begin{itemize}
  \item \textbf{\texttt{get\_next\_level}}:接收页表指针、索引和分配标志(\texttt{alloc})。
        若对应页表项无效且分配标志已启用,则分配新的页表页,
        安装非叶子页表项(有效但无R/W/X/A/D位),并返回下一级页表指针。
        若页表项已有效,则将其物理页号(PPN)转换为(恒等映射的)虚拟地址,
        作为下一级页表指针返回。
  \item \textbf{\texttt{vmm\_map}}(建立映射):
        \begin{itemize}
          \item 要求虚拟地址和物理地址均按页对齐。
          \item 遍历L2$\rightarrow$L1$\rightarrow$L0页表,按需分配中间层级页表。
          \item 将抽象的VMM标志位(\texttt{VMM\_P\_*})转换为Sv39页表项标志位(V/R/W/X/U/A/D)。
                目前\texttt{VMM\_P\_RW}隐含R/W/X权限(以便代码页和数据页无需单独设置执行权限),
                且主动设置A/D位(访问位和脏位)。
          \item 在L0层级安装叶子页表项,包含目标物理地址和权限标志。
        \end{itemize}
  \item \textbf{\texttt{vmm\_map\_page}}(映射新物理页):
        \begin{itemize}
          \item 通过\texttt{kalloc}分配新的物理页并清零,
                调用\texttt{vmm\_map}将其映射到指定虚拟地址。
        \end{itemize}
  \item \textbf{\texttt{vmm\_unmap}}(解除映射):
        \begin{itemize}
          \item 遍历页表(不分配中间层级)。
          \item 若找到有效叶子页表项,则清除该页表项;
                当\texttt{free\_phys}为非零时,将底层物理页归还给分配器。
          \item 目前不回收已空的中间页表(以简化实现,代价是少量内存开销)。
        \end{itemize}
  \item \textbf{\texttt{vmm\_translate}}(地址转换):
        \begin{itemize}
          \item 对指定虚拟地址执行软件页表遍历,
                若找到有效页表项,则返回对应的物理地址指针
                (物理页号PPN与原页偏移量组合)。
          \item 主要用于调试和内置自测试。
        \end{itemize}
\end{itemize}
额外辅助函数\texttt{vmm\_get\_page\_directory}、
\texttt{vmm\_set\_page\_directory}和\texttt{vmm\_get\_pd\_phys}
用于暴露当前内核根页表,为未来的进程独立地址空间奠定基础
(当前系统仍使用单一全局Sv39页表)。

\paragraph{Sv39架构下的用户堆分配}
用户堆基于Sv39映射机制实现,布局设计简洁:
\begin{itemize}
  \item 每个进程分配独立的堆区域,起始地址为
        \(\text{HEAP\_USER\_BASE} + \text{pid} \times \text{PER\_PROC\_HEAP}\),
        其中\texttt{HEAP\_USER\_BASE}为\texttt{0x80400000},
        \texttt{PER\_PROC\_HEAP}为8KB(每个进程堆大小)。
  \item \texttt{syscall.c}中的系统调用实现\texttt{SYS\_SBRK}(\texttt{sys\_sbrk})
        通过调用\texttt{vmm\_map\_page}为每个新增4KB页扩展堆区域,
        始终使用\texttt{VMM\_P\_RW | VMM\_P\_USER}标志,
        确保用户代码可读写新映射的内存。
  \item 全局Sv39页表与进程独立堆窗口的组合,
        实现了进程级连续虚拟堆空间,同时保持整体设计简洁
        (暂未支持进程专属\texttt{satp}寄存器切换)。
\end{itemize}

\subsection{进程管理与调度 (proc/)}

进程管理在 \texttt{proc/proc.c} 中实现。核心数据结构是进程控制块 (PCB),包含:
\begin{itemize}
  \item 进程标识符 (PID) 和父进程 PID (PPID)。
  \item 调度状态 (READY, RUNNING, BLOCKED, TERMINATED) 和优先级。
  \item 内核栈顶地址和保存的寄存器上下文 (\texttt{RegState})。
  \item 每进程堆基址 \texttt{brk\_base} 和大小 \texttt{brk\_size}。
  \item 用于调试和 Shell 输出的进程名称。
\end{itemize}

\begin{longtable}{|l|l|p{9cm}|}
  \caption{进程控制块(ProcessControlBlock, PCB)结构体定义}
  \label{tab:pcb_cn} \\
  \hline
  \textbf{成员名} & \textbf{数据类型} & \textbf{说明} \\
  \hline
  \endfirsthead  

  \caption[]{进程控制块(ProcessControlBlock, PCB)结构体定义(续表)} \\
  \hline
  \textbf{成员名} & \textbf{数据类型} & \textbf{说明} \\
  \hline
  \endhead  

  \hline
  \multicolumn{3}{r}{下一页续表} \\
  \endfoot  

  \hline
  \endlastfoot  

  pid            & int         & 进程唯一标识符 \\
  \hline
  pstat          & ProcState   & 进程运行状态(如就绪、运行、阻塞等) \\
  \hline
  name[20]       & char        & 进程名称(最大支持20个字符) \\
  \hline
  prior          & int         & 进程优先级(值越小,优先级越高) \\
  \hline
  entrypoint     & uint64\_t    & 进程入口点(起始指令地址) \\
  \hline
  stacktop       & uint64\_t    & 进程栈顶的虚拟地址 \\
  \hline
  ppid           & int         & 父进程ID(内核/初始化进程的父进程ID为0) \\
  \hline
  brk\_base      & void *      & 程序中断点基地址(堆起始地址) \\
  \hline
  brk\_size      & uint64\_t    & 已分配堆空间大小(单位:字节) \\
  \hline
  cpu\_time      & uint64\_t    & 进程消耗的总CPU时间 \\
  \hline
  remain\_time   & uint64\_t    & 进程调度剩余时间片 \\
  \hline
  arriv\_time    & uint64\_t    & 进程进入系统的到达时间 \\
  \hline
  regstat        & RegState    & 用于上下文切换的保存寄存器状态 \\
  \hline
  next           & PCB *       & 用于进程队列管理的链表指针 \\
  \hline
\end{longtable}

全局变量跟踪当前进程、Idle 进程、就绪队列以及阻塞和僵尸进程的链表。

\paragraph{Idle 进程与调度器初始化}
\texttt{scheduler\_init} 初始化就绪队列并创建一个 PID 为 0 的 \emph{idle} 进程,
拥有自己的内核栈和入口函数 \texttt{idle\_entry}。Idle 进程只是简单地开启中断并在循环中执行
\texttt{wfi},出让 CPU 直到定时器中断到达。

\paragraph{进程创建}
存在两条创建路径:
\begin{itemize}
  \item \textbf{\texttt{proc\_create}} 由内核用于创建初始 Shell 进程。
  它分配一个 PCB 和内核栈,初始化寄存器状态,以便从第一次上下文切换返回时跳转到给定的入口点,
  并将进程加入就绪队列。
  \item \textbf{\texttt{proc\_fork}} 实现 \texttt{fork} 系统调用。它克隆父进程的
   PCB 和内核栈,调整子进程的栈指针以镜像父进程的栈帧布局,将子进程的 \texttt{a0}
   寄存器设为 0(使子进程看到 \texttt{fork()} 返回 0),并使用 \
   texttt{vmm\_map\_page} 和 \texttt{memcpy} 深度复制用户堆。
   随后子进程被作为 READY 加入队列。
\end{itemize}

\paragraph{退出、等待与僵尸进程}
当进程调用 \texttt{exit} 时,\texttt{proc\_exit} 将其标记为 TERMINATED,
链接到 \texttt{zombie\_list},如果父进程阻塞在 \texttt{wait} 中,则唤醒父进程。
\texttt{proc\_wait\_and\_reap} 在 \texttt{zombie\_list} 中搜索调用进程的子进程,
释放其内核栈,通过 \texttt{vmm\_unmap} 取消映射其堆页面,释放 PCB,并返回子进程的 PID。
\texttt{ppid == 0} 的孤儿僵尸进程最终会被 \texttt{zombies\_free} 回收,该函数由调度器择机调用。

\paragraph{调度算法}
\texttt{schedule} 函数提供了一个简单的轮转 (Round-robin) 风格调度器:
\begin{itemize}
  \item 它从 \texttt{ready\_queue} 中取出下一个 READY 进程。如果没有可用的进程,
  且当前进程仍为 RUNNING 且非 Idle 进程,则当前进程继续运行。否则,调度器选择 Idle 进程。
  \item 正在运行的进程(如果有)被降级为 READY 并重新入队(Idle 进程除外)。
  TERMINATED 进程保留在僵尸链表中。
  \item 通过汇编例程 \texttt{switch\_context} 执行上下文切换,
  保存旧的 \texttt{RegState} 并恢复新的。
  \item 从上下文切换返回后,调度器调用 \texttt{zombies\_free} 清理任何孤儿僵尸进程。
\end{itemize}
调度器由周期性的机器定时器中断驱动,中断处理程序会重新编程 CLINT 定时器并从 Trap
处理程序调用 \texttt{schedule}。

\paragraph{其他进程操作}
进程子系统还提供:
\begin{itemize}
  \item \texttt{proc\_dump}: 打印所有进程的快照(由 \texttt{ps} 系统调用使用)。
  \item \texttt{proc\_suspend\_current}: 将当前进程移入阻塞链表并 \texttt{schedule}
  另一个进程;由 Shell 的 \texttt{bg} 命令使用。
  \item \texttt{proc\_kill}: 按 PID 杀死进程,搜索就绪、阻塞和僵尸链表,立即释放资源。
  \item \texttt{proc\_shutdown\_all}: 用于系统关机期间,释放就绪、阻塞和僵尸链表中的所有非 Idle 进程。
\end{itemize}

\subsection{文件系统与块设备 (fs/, fs/blk.c)}

文件系统支持构建在 VirtIO 块驱动程序和极简的基于 inode 的文件系统之上。

\paragraph{VirtIO 块驱动}
\texttt{fs/blk.c} 中的驱动程序探测 MMIO 范围内的 VirtIO 设备,寻找 \texttt{DEVICE\_ID == 2}
(块设备)的设备。它通过配置适当的队列寄存器支持 VirtIO MMIO版本 1 和版本 2。

每个 I/O 请求使用一个包含三个描述符的小型 virtqueue:
\begin{enumerate}
  \item 读写操作和目标扇区的请求头。
  \item 一个 512 字节扇区的数据缓冲区。
  \item 由设备写入的状态字节。
\end{enumerate}
通过更新可用环并通过 MMIO \texttt{QUEUE\_NOTIFY} 寄存器通知设备来提交请求。
完成检测通过轮询已用环索引实现;设备中断由 \texttt{blk\_intr} 通过 Trap 和 PLIC 代码处理。

公共接口导出 \texttt{blk\_read\_sector} 和 \texttt{blk\_write\_sector},
专供文件系统层使用。

\paragraph{磁盘布局与 Inode}
\texttt{fs/fs.c} 和 \texttt{fs/fs.h} 中的文件系统在 64 KB 磁盘镜像上使用非常小的固定大小布局:
\begin{itemize}
  \item 块 0: 超级块 (\texttt{struct superblock})。
  \item 块 1--4: Inode 表 (\texttt{struct dinode})。
  \item 块 5: 空闲块位图。
  \item 块 6--127: 数据块。
\end{itemize}
每个 Inode (\texttt{struct dinode}) 包含类型(空闲、普通文件、目录)、链接计数、文件大小、
一个包含 10 个直接块地址的数组和一个间接块指针。间接块保存一个数据块地址数组,
因此每个文件的最大块数为 \(\text{MAXFILE} = \text{NDIRECT} + \text{NINDIRECT}\)。

\paragraph{块与 Inode 管理}
内部辅助函数包括:
\begin{itemize}
  \item \texttt{b\_read} / \texttt{b\_write}: 针对 512 字节扇区的 VirtIO 驱动包装器。
  \item \texttt{balloc} / \texttt{b\_free}: 通过操作位图块中的位图来分配和释放数据块。
  \item \texttt{read\_dinode} / \texttt{write\_dinode}: 根据 Inode 编号加载和更新单个 Inode。
  \item \texttt{bmap}: 将文件块索引映射到物理块号,按需分配直接或间接块。
\end{itemize}
高层操作 \texttt{inode\_read} 和 \texttt{inode\_write} 实现了跨越可能不连续的磁盘块的稀疏块读写。

\paragraph{目录与命名空间}
当前文件系统实现了一个单一的扁平根目录,由 \texttt{sb.root\_inum} 标识。目录项
(\texttt{struct dirent}) 将文件名映射到 Inode 编号。辅助函数包括:
\begin{itemize}
  \item \texttt{dir\_lookup}: 在根目录中按名称查找 Inode 编号。
  \item \texttt{dir\_add}: 追加新的目录项。
  \item \texttt{dir\_remove\_inum}: 按 Inode 编号移除目录项。
\end{itemize}
\texttt{fs\_format} 初始化一个新的文件系统:它清除 Inode 和位图块,写入新的超级块,
创建根目录 Inode,并可选择在根目录中创建一个包含编译进内核的顶级项目 README 内容的
 \texttt{README.md} 文件。

\paragraph{文件操作与 FD 表}
在运行时,文件系统层维护一个内存表,最多包含 \texttt{FS\_MAX\_FILES} 个打开的文件
(\texttt{FSFileDesc})。返回给用户空间的文件描述符从 \texttt{FS\_FD\_BASE} 开始,
以便描述符 0, 1, 2 可以保留用于标准输入/输出/错误。

公共 API 包括:
\begin{itemize}
  \item \texttt{fs\_init}: 初始化超级块和 FD 表,如果魔数无效则格式化新文件系统。
  \item \texttt{fs\_create}: 在根目录中创建一个新的普通文件并打开它。
  \item \texttt{fs\_open}: 按名称打开现有文件。
  \item \texttt{fs\_read} / \texttt{fs\_write}: 对打开的文件执行基于偏移量的读写,更新文件偏移量。
  \item \texttt{fs\_close}: 释放文件描述符槽位。
  \item \texttt{fs\_unlink}: 按名称移除文件,释放其数据和间接块,并清除其 Inode 和目录项。
  \item \texttt{fs\_trunc}: 将文件大小设置为零而不释放其块(简单的截断语义)。
  \item \texttt{fs\_list\_root}: 将根目录项枚举到调用者提供的缓冲区中;用于实现 \texttt{ls}。
\end{itemize}

\subsection{Trap 与中断处理 (trap/)}

Trap 子系统由汇编向量入口和 C 语言处理程序组成。

\paragraph{Trap 入口}
\texttt{trap/trapentry.S} 中的入口点 \texttt{trap\_vector\_entry} 由 \texttt{trap\_init}
安装到 \texttt{mtvec} 中。它:
\begin{enumerate}
  \item 在栈上分配 128 字节作为 Trap 帧。
  \item 将部分通用寄存器 (\texttt{ra}, \texttt{t0--t2}, \texttt{a0--a7}) 保存到该帧中。
  \item 将 Trap 帧指针作为 \texttt{a0} 传递给 C 处理程序 \texttt{trap\_handler\_c}。
  \item 恢复保存的寄存器并执行 \texttt{mret} 返回到 \texttt{mepc} 中存储的地址。
\end{enumerate}

\paragraph{C 级 Trap 处理程序}
核心逻辑在 \texttt{trap/trap.c} 中。它读取 \texttt{mcause}, \texttt{mepc},
\texttt{mtval}, 和 \texttt{mstatus},解码事件是异常还是中断,并进行相应分发。

对于 \textbf{环境调用 (Environment call)} 异常(来自用户或 Machine 模式的 ecall),处理程序:
\begin{itemize}
  \item 从保存的 \texttt{a7} (Trap 帧中的 \texttt{tf[11]}) 提取系统调用号,
  从 \texttt{a0--a5} 提取最多六个参数。
  \item 从 Trap 帧更新当前进程的 \texttt{RegState} 快照,以便随后的 \texttt{fork}
  能看到最新的用户上下文。
  \item 对于 \texttt{SYS\_EXEC},执行特殊处理:它查找程序名称以获取入口点,
  更新 \texttt{mepc} 到新入口,并为 \texttt{argc} 和 \texttt{argv} 配置
   \texttt{a0} 和 \texttt{a1}。此路径绕过通用分发器。
  \item 对于所有其他系统调用,调用 \texttt{syscall\_dispatch},将返回值写回 Trap 帧中的
  \texttt{a0},并将 \texttt{mepc} 增加 4 以跳过 \texttt{ecall} 指令。
\end{itemize}

对于 \textbf{ecall 以外的异常},处理程序记录诊断信息(启用 Trap 调试时),
并将该错误视为当前进程的致命错误,调用 \texttt{proc\_exit}。

对于 \textbf{中断},主要情况有:
\begin{itemize}
  \item 机器定时器中断:重新编程 CLINT 定时器以在固定间隔后再次触发,并调用 \texttt{schedule}
        执行上下文切换。
  \item 机器外部中断:通过 \texttt{plic\_claim} 从 PLIC 获取中断号,将 VirtIO 相关中断
  分发给 \texttt{blk\_intr},然后通过 \texttt{plic\_complete} 完成中断。
\end{itemize}
任何意外的 Trap 都会落入无限 \texttt{wfi} 循环,以便在调试时安全停机。

\paragraph{PLIC 设置}
\texttt{trap/plic.c} 模块通过以下方式配置 QEMU 的平台级中断控制器:
\begin{itemize}
  \item 为 IRQ 1--8 (VirtIO 设备) 设置非零优先级。
  \item 为 hart 0 启用这些 IRQ。
  \item 将中断优先级阈值设置为零。
\end{itemize}
辅助函数 \texttt{plic\_claim} 和 \texttt{plic\_complete} 与 PLIC claim/complete
寄存器交互。

\subsection{系统调用实现 (syscall/)}

系统调用在 \texttt{kernel/syscall/syscall.h} 中定义 ID,并在
\texttt{kernel/syscall/syscall.c} 中实现。分发器 \texttt{syscall\_dispatch}
使用 switch 语句调用正确的内核函数。

\subsubsection{系统调用表}

表~\ref{tab:syscalls} 总结了当前实现的系统调用。

\begin{longtable}{|c|l|p{0.55\linewidth}|}
\caption{Lrix 系统调用接口。}
\label{tab:syscalls} \\

\hline
ID & 名称 & 描述 \\
\hline
\endhead
1  & SYS\_EXIT     & 终止当前进程并将其转为僵尸状态;如果父进程正在等待,则唤醒它。 \\
2  & SYS\_GETPID   & 返回调用进程的 PID。 \\
3  & SYS\_FORK     & 通过复制调用者的 PCB、内核栈和用户堆来创建子进程。 \\
4  & SYS\_WAIT     & 阻塞直到一个子进程终止;回收其资源并返回其 PID。 \\
5  & SYS\_SBRK     & 使用 VMM 按请求的增量增长调用进程的堆。 \\
6  & SYS\_SLEEP    & 基于 CLINT \texttt{mtime} 滴答的忙等待睡眠,使用 \texttt{wfi}。 \\
7  & SYS\_KILL     & 按 PID 杀死进程;如果找到,立即释放其资源。 \\
8  & SYS\_UPTIME   & 返回当前 CLINT \texttt{mtime} 值作为单调递增的滴答计数器。 \\
9  & SYS\_WRITE    & 通过 \texttt{printk} 向 stdout/stderr (fd 1/2) 写字节,或向文件系统支持的描述符写字节。 \\
10 & SYS\_OPEN     & 按名称打开文件,可选创建标志;返回文件系统文件描述符。 \\
11 & SYS\_READ     & 从文件系统支持的描述符读取字节到用户缓冲区。 \\
12 & SYS\_CLOSE    & 关闭文件系统支持的描述符。 \\
13 & SYS\_LS       & 将根目录条目列出到用户提供的 \texttt{dirent} 数组中。 \\
14 & SYS\_GETC     & 从 UART 控制台阻塞读取单个字符。 \\
15 & SYS\_UNLINK   & 移除根目录中的文件,释放其块并清除其 Inode。 \\
16 & SYS\_EXEC     & 用命名的用户程序替换当前进程镜像;在 Trap 处理程序中特殊处理。 \\
17 & SYS\_TRUNC    & 按名称截断文件,将其大小设为零但保留其块分配。 \\
18 & SYS\_PS       & 通过 \texttt{proc\_dump} 转储内核进程列表。 \\
19 & SYS\_SHUTDOWN & 关闭系统:禁用中断,释放所有进程,并通过 \texttt{wfi} 停机。 \\
20 & SYS\_SUSPEND  & 将当前进程移至阻塞链表并调度另一个进程;由后台 worker 使用。 \\
\hline
\end{longtable}

\subsubsection{Exec 支持}

Exec 风格的程序替换以简单的方式实现。一个小型的静态 \texttt{exec\_table}
将程序名称映射到链接进内核镜像的函数入口点,例如:
\begin{itemize}
  \item \texttt{"sh"} \(\rightarrow\) \texttt{user\_shell}.
\end{itemize}
辅助函数 \texttt{sys\_exec\_lookup} 在处理 \texttt{SYS\_EXEC} 时由 Trap
处理程序调用。成功时,处理程序更新 \texttt{mepc} 到新的用户入口点并重新初始化 \texttt{a0}
和 \texttt{a1} (\texttt{argc} 和 \texttt{argv})。

\section{用户空间与 Shell (usr/)}
用户空间代码位于 \texttt{usr/} 下。所有程序使用公共头文件 \texttt{user.h},
它导入内核可见的定义(文件系统类型、字符串辅助函数、系统调用号)并声明建立在
\texttt{sys\_call.c} 实现的通用 \texttt{sys\_call3} 函数之上的瘦系统调用封装。

\subsection{系统调用封装}

每个用户可见的函数如 \texttt{sys\_write}, \texttt{sys\_open}, \texttt{sys\_fork},
或 \texttt{sys\_shutdown} 都是一个小型的封装,它在寄存器 \texttt{a0--a2} 中设置参数,
在 \texttt{a7} 中设置系统调用号,并执行 \texttt{ecall}。内核的 Trap 处理程序随后
分发到相应的系统调用实现。

\subsection{Shell 概览}

主要的用户程序是在 \texttt{usr/shell.c} 中实现的交互式 Shell。它作为一个常规进程运行,
并仅通过系统调用与内核交互。

Shell 的架构包括:
\begin{itemize}
  \item 一个简单的行编辑器,使用 \texttt{sys\_getc} 读取字符,本地处理退格键,并通过
  \texttt{sys\_write} 回显输入。
  \item 一个标记化例程,将命令行分割成最多八个以空格分隔的参数 (\texttt{argv})。
  \item 一组直接用 C 实现的内置命令,无需衍生外部程序。
  \item 通过临时文件名 (\texttt{\_\_pipe.tmp}) 对“管道”的极简支持,用于在简单场景下
  将一个命令的输出传递给另一个命令。
\end{itemize}

\subsection{支持的 Shell 命令}

Shell 目前支持以下内置命令:

\begin{itemize}
  \item \textbf{\texttt{ls}}: 通过调用 \texttt{sys\_ls} 并打印每个返回的 \texttt{dirent}
                              名称来列出根目录中的文件。
  \item \textbf{\texttt{cat FILE}}: 使用 \texttt{sys\_open} 打开文件,使用
                                    \texttt{sys\_read} 读取内容,并写入标准输出。
                                    当在管道模式下不带参数调用时,它从临时管道文件读取。
  \item \textbf{\texttt{read FILE}}: \texttt{cat FILE} 的别名。
  \item \textbf{\texttt{echo ARGS...}}: 打印以空格分隔并以换行符结尾的参数。
  \item \textbf{\texttt{touch F}}: 如果文件不存在则创建空文件,
                                   尝试 \texttt{sys\_open(F, 0)},
                                  失败则 \texttt{sys\_open(F, 1)}。
  \item \textbf{\texttt{rm F}}: 使用 \texttt{sys\_unlink} 移除文件;
                                失败时打印错误消息。
  \item \textbf{\texttt{mv A B}}: 移动(重命名)文件,通过将数据从 A 复制到 B 然后
                                  unlink A。如果 B 存在,则先 unlink 并重新创建。
  \item \textbf{\texttt{pwd}}: 打印当前目录,由于文件系统目前是扁平的,始终为 \texttt{/}。
  \item \textbf{\texttt{mkdir D}}, \textbf{\texttt{rmdir D}}: 打印信息性消息,
                                                              表明不支持分层目录。
  \item \textbf{\texttt{write F TEXT...}}: (重)写文件。该命令将文件名后的所有参数连接
                                          成一个空格分隔的字符串并写入文件,先通过
                                          \texttt{sys\_trunc} 截断文件。当与管道一起使用时,
                                          也可以将临时管道文件的内容写入目标文件。
  \item \textbf{\texttt{fork}}: \texttt{fork} 系统调用的演示。父进程打印子进程的 PID
                                然后调用 \texttt{sys\_wait} 进行同步。子进程打印消息并退出。
  \item \textbf{\texttt{bg}}: 创建一个简单的后台 worker 进程,通过 fork 并在子进程中
                              调用 \texttt{sys\_suspend},将其移入阻塞链表而不消耗 CPU。
  \item \textbf{\texttt{kill PID}}: 通过 \texttt{sys\_kill} 按 PID 杀死进程。
  \item \textbf{\texttt{ps}}: 通过调用 \texttt{SYS\_PS} 系统调用列出进程,
                              该调用在内核中调用 \texttt{proc\_dump}。
  \item \textbf{\texttt{help}}: 打印所有内置命令的摘要及每个命令的简短描述。
  \item \textbf{\texttt{exit}}: 通过调用 \texttt{sys\_shutdown} 关闭系统。
                                Shell 打印有关退出 QEMU 模拟器的提示。
  \item \textbf{\texttt{halt}}: \texttt{exit} 的同义词,直接调用
                                \texttt{sys\_shutdown} 而不显示额外消息。
\end{itemize}

通过这些命令,Shell 演练了大部分内核子系统:
文件系统操作 (\texttt{ls}, \texttt{cat}, \texttt{write}, \texttt{rm}, \texttt{mv},
\texttt{touch})、进程控制 (\texttt{fork}, \texttt{bg}, \texttt{kill}, \texttt{ps})
以及关机路径 (\texttt{exit}, \texttt{halt})。

\section{总结}
Lrix 展示了从裸机 RISC-V 启动到交互式用户 Shell 的紧凑而完整的路径。
尽管体积小巧,它涵盖了许多经典的操作系统概念:基于页的物理分配器、极简的虚拟内存抽象、
带有阻塞和僵尸状态的轮转进程调度、基于 VirtIO 块存储的简单 inode 文件系统、
绑定到 RISC-V CSR 和 PLIC 的 Trap 与中断子系统,以及供用户空间程序使用的系统调用接口。

由于设计特意保持直观且代码库完全自包含,Lrix 可以作为实验和扩展的良好基础。但是我还是得
强调 Lrix 只运行在 Machine Mode下。

\printbibliography

\end{document}
